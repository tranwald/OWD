% Preambuła
\documentclass[a4paper,10pt,fleqn]{article}
\usepackage{fullpage}
\usepackage{lscape}	
\usepackage{amsmath}
\usepackage{amsfonts}
\usepackage[polish]{babel}
\usepackage[OT4]{fontenc}
\usepackage[utf8]{inputenc}
\usepackage{enumitem}
\usepackage{rotating}

% Część główna
\title{Optymalizacja Wspomagania Decyzji\\Sprawozdanie z projektu}
\author{Andrzej Smyk}
\begin{document}
\maketitle
	\section{Skrócona treść zadania}
		{\bf OD \hfill Dane nr 20\\\hfill}
		\\\hfill
		Zakłady chemiczne produkują 3 rodzaje produktów: P1, P2, P3. Wykorzystuje się do tego dwa rodzaje surowca S1 i S2 dostępnego w ilościach,
		odpowiednio 9000 12000 ton dziennie. Ceny Surowca S1 i S2 wynoszą, odpowiednio 140 i 110 zł za tonę.

		Surowce są poddawane wstępnej obróbce w przygotowalni o całkowitej dziennej przepustowości 14722 ton. 
		w wyniku tego powstają trzy półprodukty: D1,D2 i D3. Ilość poszczególnych półproduktów w zależności 
		od surowca kształtuje się następująco 

		\begin{center}
		    \begin{tabular}{ | c | c | c | c |}
		    \hline
		    Surowiec & D1 & D2 & D3 \\ \hline
			S1 & 0,2 & 0,5 & 0,3 \\ \hline
			S2 & 0,1 & 0,7 & 0,2 \\ \hline
		    \end{tabular}
		\end{center}

		D1 może być bezpośrednio użyty do produkcji P1 i P3. Natomiast D2 i D3 mogą być wykorzystane
		do bezpośredniej produkcji P2 lub przeznaczone do dalszej obróbki w zakładzie uwodornienia. Koszt przygotowania surowca zależny jest od wielkości 
		dziennego przerobu danego surowca i jest przedstawiony w tabeli poniżej

		\begin{center}
		    \begin{tabular}{ | l | l | l |}
		    \hline
		    & Dzienny przerób & Koszt przygotowania \\ \hline
			S1 & od 1860 ton włącznie & 12zł/tonę \\ \hline
			& od 1860 do 5398 włącznie & 1860*19zł+14zł za każdą tonę powyżej progu 1860 ton \\ \hline
			& od 5398 & 1860*19zł+(5398-1860)*14zł+10zł za każdą tonę powyżej progu 5398 ton \\ \hline
			S2 & od 3694 ton włącznie & 12zł/tonę \\ \hline
			& od 3694 do 8013 włącznie & 3694*12zł+16zł za każdą tonę powyżej progu 3694 ton \\ \hline
			& od 8013 & 3694*12zł+(8013-3694)*16zł+20zł za każdą tonę powyżej progu 8013 ton \\ \hline
		    \end{tabular}
		\end{center}

		Zakład uwodornienia o przepustowości dziennej 5454 ton wytwarza półprodukty K1 i K2. Ilość poszczególnych półproduktów w zależności od surowca kształtuje się następująco

		\begin{center}
		    \begin{tabular}{ | c | c | c |}
		    \hline
		    Półprodukt & D2 & D3 \\ \hline
			K1 & 0,1 & 0,4 \\ \hline
			K2 & 0,9 & 0,6 \\ \hline
		    \end{tabular}
		\end{center}

		Półprodukty Ki i K2 mogą być użyte do produkcji P1 lub P2, ale nie mogą być użyte do wytworzenia P3. Jeśli 
		kład uwodornienia pracuje, dzienny koszt jego pracy (niezależnie od ilości przetworzonych produktów) wynosi
		13 tys. zł. Jeśli nie pracuje, dzienny koszt jest zerowy. Cena sprzedaży hurtowej wynosi: 176 zł/tonę P1, 140 zł/tonę i
		109 zł/tonę P3. Zawarte umowy wymagają dostarczenia co najmniej 3495 ton każdego produktu.

		Dany produkt finalny jest wytwarzany z półproduktów D1, D2, D3, K1 lub K2 bezpośrednio bez utraty masy, jednak zgodnie
		z zasadami opisanymi wcześniej.

		Należy zaprojektować system wspomagania decyzji, w którym są 4 kryteria: całkowity koszt produkcji oraz względne niedobory
		każdego produktu. System powinien wykorzystywać metodę punktu odniesienia do agregacji kryteriów.

	\section{Model}
		W stworzonym modelu, zgodnie z tym co zostało opisane w treśici zadania, wszystkie pólprodukty i produkty wytwarzane są bez utraty masy, tj. ilość półproduktu/surowca wejściowego musi byc równa ilości półproduktu/produktu wyjściowego.
		\begin{itemize}
			\item ;
		\end{itemize}

		\subsection{Zmienne decyzyjne}

		Zmienne ilościowe (w tonach):
		\begin{description}[align=left,labelwidth=1.5cm]
			\item[$S_i$] ilości surowca $S$, $i = 1,2$
			\item[$s_{ij}$] ilości surowca $S$ dla różnych progów kosztu przygotowania, gdzie $i$ to numer surowca, $j$ numer progu
			\item[$D_i$] ilości półproduktów $D$, $i = 1, 2, 3$
			\item[$K_i$] ilość półproduktów $K$, $i = 1, 2$
			\item[$DK_2$] ilość półproduktu $D2$ wykorzystanego w procesie uwodornienia
			\item[$DK_3$] ilość półproduktu $D3$ wykorzystanego w procesie uwodornienia
			\item[$DP_{ij}$] ilość półproduktu $D_i$ wykorzystanego w produkcji $P_j$, \\$i=1, 2, 3 \quad j=1, 2, 3$
	
			\item[$KP_{ij}$] ilość półproduktu $K_i$ wykorzystanego w produkcji $P_i$, \\$i=1, 2\quad j=1, 2$
		\end{description}

		Zmienne binarne (flagi dodatności):
		\begin{description}[align=left,labelwidth=1.5cm]
			\item[$w$] flaga pracy zakładu uwodorowienia
			\item[$u_{ij}$] flaga progów kosztu przygotowania surowców $S_1$ i $S_2$, \\$i = 1, 2 \quad j = 1, 2$
		\end{description}

		\subsection{Ograniczenia}

		Dostępność surowców:
		\begin{subequations}
			\begin{equation}
				0 \leq S_1 \leq 9000
			\end{equation}
			\begin{equation}
				0 \leq S_2 \leq 12000
			\end{equation}
		\end{subequations}

		Całkowita dzienna przepustowość przygotowalni:
		\begin{equation}
			D_1 + D_2 + D_3 \leq 14722
		\end{equation}

		Ilości półproduktów $D_1$, $D_2$, $D_3$ w zależności od surowca:
		\begin{subequations}
			\begin{equation}
				 D_1 = 0.2S_1 + 0.1S_2
			\end{equation}			
			\begin{equation}
				D_2 = 0.5S_1 + 0.7S_2
			\end{equation}
			\begin{equation}
				D_3 = 0.3S_1 + 0.2S_2
			\end{equation}
		\end{subequations}

		Całkowita dzienna przepustowość zakładu uwodorowienia:
		\begin{equation}
			K_1 + K_2 \leq 5454
		\end{equation}

		Półprodukty $D_2$ i $D_3$ mogą zostać wykorzystane do bezpośredniej produkcji $P2$ lub przeznaczone do dalszej obróbki w zakładzie uwodorowienia:
		\begin{subequations}
			\begin{equation}
				D_2 = DP_{22} + DK_2
			\end{equation}
			\begin{equation}
				D_3 = DP_{32} + DK_3
			\end{equation}
		\end{subequations}

		Ilości poszczególnych półproduktów $K_1$ i $K_2$ w zależności od wykorzystanych półproduktów $D_2$ oraz $D_3$:
		\begin{subequations}
			\begin{equation}
				K_1 = 0.1DK_2 + 0.4DK_3
			\end{equation}
			\begin{equation}
				K_2 = 0.9DK_2 + 0.6DK_3
			\end{equation}
		\end{subequations}

		Do produkcji $P1$ mogą zostać wykorzystane półprodukty $D_1$, $K_1$ oraz $K_2$:
		\begin{equation}
			P_1 = DP_{11} + KP_{11} + KP_{21}
		\end{equation}

		Do produkcji $P2$ mogą zostać wykorzystane półprodukty $D_2$, $D_3$, $K_1$ oraz $K_2$:
		\begin{equation}
			P_2 = DP_{22} + DP_{32} + KP_{12} + KP_{22} 
		\end{equation}

		Do produkcji $P_3$ może zostać wykorzystany połprodukt $D_1$:
		\begin{equation}
			P_3 = DP_{13}
		\end{equation}

		Półprodukt $D_1$ może zostać wykorzystany do produkcji $P_1$ oraz $P_3$:
		\begin{equation}
			D_1 = DP_{11} + DP_{13}
		\end{equation}

		Półprodukt $K_1$ może zostać wykorzystany do produkcji $P_1$ oraz $P_2$:
		\begin{equation}
			K_1 = KP_{11} + KP_{12}
		\end{equation}

		Półprodukt $K2$ może zostać wykorzystany do produkcji $P_1$ oraz $P_2$:
		\begin{equation}
			K_2 = KP_{21} + KP_{22}
		\end{equation}

		Jeśli zakład uwodorowienia pracuje to dzienny koszt jego pracy wynosi 13000 zł., w innym wypadku 0:
		\begin{subequations}
			\begin{equation}
				0 \leq K_1 + K_2 \leq 5454w
			\end{equation}
			\begin{equation}
				0 \leq w \leq 1 \qquad w \in Z
			\end{equation}
		\end{subequations}

		Koszt przygotowania surowca zależny jest od wielkości dziennego przerobu danego surowca:
		\begin{itemize}
			\item $S_1$ (koszty malejący):
				\begin{subequations}
					\begin{equation}
						 S_{1} = s_{11} + s_{12} + s_{13}
					\end{equation}	
					\begin{equation}
						1860u_{11} \leq s_{11} \leq 1860
					\end{equation}
					\begin{equation}
						(5398-1860)u_{12} \leq s_{12} \leq (5398-1860)u_{11}
					\end{equation}
					\begin{equation}
						0 \leq s_{13} \leq (9000-5398)u_{12}
					\end{equation}
					\begin{equation}
						0 \leq u_{1i} \leq 1 \quad u_{i} \in Z \quad i = 1, 2
					\end{equation}
				\end{subequations}	
			\item  $S_2$ (koszty rosnące):
				\begin{subequations}
					\begin{equation}
						 S_2 = s_{21} + s_{22} + s_{23}	
					\end{equation}	
					\begin{equation}
						3694u_{21} \leq s_{11} \leq 3694
					\end{equation}
					\begin{equation}
						( 8013 - 3694)u_{22} \leq s_{12} \leq ( 8013 - 3694)u_{21}
					\end{equation}
					\begin{equation}
						0 \leq s_{13} \leq ( 8013 - 3694)u_{22}
					\end{equation}
					\begin{equation}
						0 \leq u_{2i} \leq 1 \quad u_{i} \in Z \quad i = 1, 2
					\end{equation}
				\end{subequations}	
		\end{itemize}
		
		\subsection{Kryteria}
		Całkowity koszt produkcji składa się z:
		\begin{itemize}
			\item kosztu surowców $S_1$ oraz $S_2$:
				\begin{equation}
					c_S = 140S_1 + 110S_2
				\end{equation}
			\item stałego kosztu pracy zakładu uwodorowienia:
				\begin{equation}
					c_w = 13000*w
				\end{equation}
			\item kosztu przygotowania surowca $S_1$:
				\begin{equation}
					c_{s1} = 12s_{11} + 8s_{12} + 5s_{13}
				\end{equation}
			\item kosztu przygotowania surowca $S_2$:
				\begin{equation}
					c_{s2} = 12s_{21} + 16s_{22} + 21s_{23}
				\end{equation}
		\end{itemize}

		Po zsumowaniu wszystkich składników, całkowity koszt wygląda następująco:
		\begin{equation}
			\begin{split}
				y_1 &= c_w + c_{s1} + c_{s2} + c_s \\
			& = 13000w + 12s_{11} + 8s_{12} + 5s_{13} + 12s_{21} + 16s_{22} + 21s_{23} + 140S_1 + 110S_2
			\end{split}
		\end{equation}

		Względny niedobór produktów $P_1$, $P_2$, $P_3$:
		\begin{subequations}
			\begin{equation}
				y_{2}= \frac{3495-P_1}{3495}
			\end{equation}
			\begin{equation}
				y_{3}= \frac{3495-P_2}{3495}
			\end{equation}
			\begin{equation}
				y_{4}= \frac{3495-P_3}{3495}
			\end{equation}
		\end{subequations}

		\subsection{Metoda punktu odniesienia}

		Jako wartość parametru regulacji $\epsilon$ jest ustalona jako $\sigma/\text{m}$, gdzie $m$ jest liczbą kryterów zaś $\sigma$ arbitralnie dobraną małą stałą. Jako czynnik $\beta$ przyjęto typową wartość $10^{-3}$. Wszystkie mnożniki $\lambda$ ustalono na $1$.
		\\\hfill
		Parametry dla metody punktu odniesienia:
		\begin{flalign*}
			& \sigma = 10^{-4} \\
			& \varepsilon = \frac{\sigma}{m}=0.00025 \\
			& \lambda_{ij} = 1 \qquad\text{dla} \quad i=1 \dots 4, \quad j=1 \dots 4  \\
			& \beta=10^{-3}
		\end{flalign*}
		
		Inżynierska aimplementacja metody punktu odniesienia:
		\begin{flalign*}
			\text{max} \quad  & v + \varepsilon\sum_{i=1}^{m}{z_i} \\
			& v \leq z_i \qquad i =1\dots m \\
			& z_i \leq \beta\lambda_i(y_i - a_i)\qquad i =1\dots m \\
			& z_i \leq \lambda_i(y_i - a_i)\qquad i =1\dots m
		\end{flalign*}

	\section{Model w AMPL}
		Wszystkie pliki potrzebne do przeprowadzenia symulacji znajdują się w katalogi \texttt{src}.
		Model został zapisany w pliku \texttt{model.mod}, zaś symulację można uruchomić za pomocą pliku \texttt{owd.run}. Wszystkie ograniczenia w modelu zawierają w nazwach odpowiednie numery konkretnych równań ze sprawozdania.

		
	\section{Symulacja}
		\begin{landscape}
			\begin{center}
			    \begin{tabular}{ | l | c | c | c | c | c | c | c | c |c |}
			    \hline
				
			    $a$ & $S$ & $D$ & $K$ & $P$ & Koszt & $RD_{1}$  & $RD_{2}$ & $RD_{3}$ & Zysk \\ \hline

			    	[[2000000, 1.0000, 1.0000, 1.0000] &  [9000, 5720] & [2372, 8504, 3844] & [1698, 3756] & [6894, 6894, 932] & 2047578 & -0.9725  & -0.9725  & 0.7333  & 232514  \\ \hline

			    	[2000000, 0.0000, 0.0000, 0.0000] &  [9000, 5720] & [2372, 8504, 3844] & [1698, 3756] & [5454, 6894, 2372] & 2047578 & -0.5605  & -0.9725  & 0.3213  & 136034 \\ \hline

				[[2000000, -3.0000, -3.0000, -3.0000] &  [9000, 5720] & [2372, 8504, 3844] & [546, 4908] & [7826, 6894, 0] & 2047578 & -1.2392  & -0.9725  & 1.0000  & 294957  \\  \hline
				
				[0, -1.633, -0.699, 0.065] &  [10990, 1940] & [4590, 4849, 3491] & [5004, 3336] & [8657, 4043, 230] & 1841761 & -0.760  & 0.177  & 0.953  & 401888 \\  \hline
				 [1800000, -1.633, -0.699, 0.065] &  [10760, 2110] & [4515, 4916, 3439] & [5013, 3342] & [8664, 4004, 202] & 1831077 & -0.762  & 0.1857  & 0.958  & 405819 \\  \hline
				 [1800000, -2, -0.699, 0.065] &  [10750, 2100] & [4510, 4905, 3435] & [5004, 3336] & [9107, 3743, 0] & 1828273 & -0.852  & 0.238  & 1  & 438426 \\  \hline
				 [1800000, -3, -0.699, 0.065] &  [11000, 1950] & [4595, 4860, 3495] & [5013, 3342] & [12950, 0, 0] & 1844461 &  -1.633  & 1  & 1  & 654889 \\  \hline
				 [1500000, -1.633, -0.699, 0.065] &  [10970, 1970] & [4585, 4867, 3488] & [5013, 3342] & [8626, 4042, 272] & 184284 & -0.754  & 0.1780  & 0.944  & 398881 \\  \hline
				 [1500000, -1, -1, -1] &  [11000, 1950] & [4595, 4860, 3495] & [5013, 3342] & [4310, 4320, 4320] & 1844461 & 0.123  & 0.121  & 0.121  & 6889\\ \hline
				 [1800000, 0, 0, 0] &  [11000, 1950] & [4595, 4860, 3495] & [5013, 3342] & [4310, 4320, 4320] & 1844461 & 0.123  & 0.121  & 0.121  & 6889 \\ \hline
				 [1800000, 0, 0.5, 0.5] &  [10970, 1970] & [4585, 4867, 3488] & [5013, 3342] & [6020, 3494, 3426] & 1842837 & -0.224  & 0.289  & 0.303  & 136807 \\ \hline
				 [1000000, -1, 0.5, 0.5] &  [10940, 1990] & [4575, 4874, 3481] & [5013, 3342] & [9232, 1857, 1841] & 1841261 & -0.877  & 0.622  & 0.625  & 377167 \\ \hline
				 [1000000, 0.5, -1, 0.5] &  [10880, 2030] & [4555, 4888, 3467] & [5013, 3342] & [2258, 8354, 2298] & 1837809 & 0.540  & -0.699  & 0.532  & -36071 \\ \hline
				 [1000000, 0.5, 0.5, -1] &  [10970, 1970] & [4585, 4867, 3488] & [5013, 3342] & [4184, 4172, 4584] & 1842861 & 0.149  & 0.151  & 0.067  & -9557\\ \hline
				 [1000000, -1, -1, 1] &  [11000, 1950] & [4595, 4860, 3495] & [5013, 3342] & [4598, 8352, 0] & 1844461 & 0.064  & -0.6986  & 1  & 178825 \\ \hline
				 [1000000, -2, 0, -1] &  [10850, 2050] & [4545, 4895, 3460] & [5013, 3342] & [9771, 1, 3128] & 1836129 & -0.987  & 0.9998  & 0.363 & 362610 \\ \hline
				 
			    \end{tabular}
			\end{center}

			Dla zadanych parametrów modelu zawsze opłaca się kierować półprodukty do uwodornienia. Największy zysk zakład uzyskuje produkując P1 kosztem niedoborów innych produktów. Najbardziej nieopłacalnym jest produkt P3.

		\end{landscape}

\end{document}